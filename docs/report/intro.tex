\chapter*{Введение}
\addcontentsline{toc}{chapter}{Введение}

В настоящее время компьютерная графика используется достаточно широко. Типичная область ее применения – это кинематография и компьютерные игры. 

На сегодняшний день большое внимание уделяется алгоритмам получения реалистичного изображения. Зачастую эти алгоритмы ресурсозатратны: чем более
качественное изображение требуется получить, тем больше времени и памяти
тратится на его синтез. Это становится проблемой при создании динамической
сцены, где на каждом временном интервале необходимо производить расчеты
заново. 

Цель данной работы --- реализовать построение трехмерной сцены и 
визуализацию погодных эффектов в сельской местности.

Чтобы достигнуть поставленной цели, требуется решить следующие задачи:
\begin{enumerate}[label=\arabic*)]
	\item описать структуру трехмерной сцены, включая объекты, из которых состоит
	сцена, и дать описание выбранных погодных явлений, которые будут
	визуализированы;
	\item провести анализ алгоритмов построения реалистичных изображений;
	\item выбрать и/или модифицировать существующие алгоритмы трехмерной
	графики, которые позволят построить реалистичные изображения;
	\item реализовать выбранные алгоритмы;
	\item разработать программное обеспечение, которое позволит отобразить
	трехмерную сцену и визуализировать погодные эффекты в сельской местности.
\end{enumerate} 